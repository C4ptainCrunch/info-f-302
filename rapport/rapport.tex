\documentclass[a4paper]{article}

\usepackage[utf8]{inputenc}
\usepackage[T1]{fontenc}
\usepackage[french]{babel}
\usepackage{fullpage}
\usepackage{hyperref}
\usepackage{amssymb}
\usepackage{upgreek}

\title{
    Orthogonal Packing\\
    \small Projet d'informatique fondamentale - INFO-F-302
}
\author{
    Nikita \bsc{Marchant} et Romain \bsc{Fontaine}\\
    ULB BA3 2015-2016
}
\date{20 mai 2016}



\begin{document}
\maketitle
\tableofcontents

\section{\'Enonc\'e}
\subsection{Définitions}

\subsection{Objectif}

\section{Questions}
\subsection{Ecrire les contraintes que soit satfisfaire $\mu$}

$\mu$ doit satisfaire les contraintes suivantes :

\begin{itemize}
  \item Tout rectanlge ne doit pas dépasser du grand rectangle :
  $$\forall i \in k,
  A(i) + \mathcal{X}(i) \leq n
  \land B(i) + \mathcal{Y}(i) \leq m
  \land A(i) \geq 0
  \land B(i) \geq 0$$

  \item Deux rectangles ne peuvent pas se superposer :
  $$\forall i, \forall j, \neq i,
  A(i) + \mathcal{X}(i) \leq  A(j)
  \lor A(i) \geq A(j) + \mathcal{X}(j)
  \lor B(i) + \mathcal{Y}(i) \leq  B(j)
  \lor B(i) \geq B(j) + \mathcal{Y}(j)
  $$
\end{itemize}

\subsection{Construire une formule $\Phi$ en FNC}
On commence par expirmer chaque contrainte en FNC :
\begin{itemize}
  \item $C_1$ : $$
\end{itemize}

\end{document}
